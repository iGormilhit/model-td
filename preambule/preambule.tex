% preambule

\documentclass[11pt,a4paper]{article}
% classe, format, police : il est possible de préciser 'oneside'(recto) ou 'twoside' (recto-verso)

\usepackage{fontspec} % utile pour une typographie avancée
	\setmainfont{Arial} % police Arial (doit être installée sur la machine hôte)
\usepackage{xunicode} % gère l'unicode (utf-8)
\usepackage{polyglossia} % gestion des documents avec plusieurs langues
\usepackage{url}
	\setmainlanguage{french} % configure la langue principale
\usepackage{csquotes} % environnement de citation, gère les guillemets français


\usepackage[a4paper]{geometry} % pour configurer les marges de la page
	\geometry{verbose,tmargin=2.5cm,bmargin=2.5cm,lmargin=3.5cm,rmargin=2.5cm}	% les marges
	\setlength{\parskip}{9pt} % espacement vertical entre les paragraphes
	\setlength{\parindent}{0pt} % intendation de la première ligne des paragraphes
\renewcommand{\baselinestretch}{1.5} % modification de l'interligne

\usepackage{titling} % récupère les métadonnées pour les réutiliser
\usepackage{enumerate} % gère les listes à puces ou numérotées
\usepackage{enumitem}
\usepackage{fancyhdr} % configuration de l'en-tête
	\pagestyle{fancy} % configuration de l'en-tête (style)
% \usepackage{latexsym} % symboles : ajoute des symboles supplémentaires (optionnel)
\usepackage{float} % figures flottantes : voir https://fr.wikibooks.org/wiki/LaTeX/%C3%89l%C3%A9ments_flottants_et_figures
\usepackage{graphicx} % permet l'ajout de toute sortes d'images (y compris .EPS)
\usepackage{titlesec} % pour gérer les titres de section
\usepackage{titletoc}
\usepackage{caption} % pour les légendes
\usepackage{setspace} % pour l'interligne
\usepackage{array} % pour les largeur de colonnes, également pour les tableaux
\usepackage{hyperref} % pour les url et les liens internes
\usepackage{setspace}
